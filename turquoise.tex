%be sure to use Xelatex (coz we're using additional fonts)

\documentclass{beamer}
\usepackage{mathptmx}
\usepackage{bibentry}
\usepackage{times}
\usepackage{ragged2e}
\usepackage{fontspec}


\newfontfamily{\Monstserrat}{Montserrat-Regular.ttf}
\title{Something bout' Olive Green, she said}
\subtitle{"Ooh, something bout leaf too...i guess"}

\author[]{Athira Rengith}
\newcommand{\college}{Vidya Academy of Science and Techonology}
\newcommand{\regno}{\small VAS17CS 039}
\newcommand{\dept}{(Dept of CSE), VAST}
\newcommand{\whitelogo}{leaf.png}
\newcommand{\bluelogo}{leaf.png}

\usetheme{turquoise}
\useoutertheme{progress}
\bibliographystyle{plain}

\begin{document}
	
\begin{frame}
\titlepage
\end{frame}

\section[Introduction]{}



\section[Abstract]{}
\begin{frame}[allowframebreaks]
\frametitle{Abstract}
With the wide applications of smart devices and mobile computing, smart home becomes
a hot issue in the household appliance industry. The controlling and interaction approach plays a key role
in users’ experience and turns into one of the most important selling points for profit growth. Consid-
ering the robustness and privacy protection, wearable devices equipped with MEMS, e.g., smartphones,
smartwatches, or smart wristbands, are thought of one of the most feasible commercial solutions for
interaction. However, the low-cost built-in MEMS sensors do not perform well in capturing finely grained
human activity directly. In this paper, we propose a method that leverages the arm constraint and historical
information recorded by MEMS sensors to estimate the maximum likelihood action in a two-phases model.
First, in the arm posture estimation phase, we leverage the kinematics model to analyze the maximum
likelihood position of users’ arms. Second, in the trajectory recognition phase, we leverage the gesture
estimation model to identify the key actions and output the instructions to devices by SVM. Our substantial
experiments show that the proposed solution can recognize eight kinds of postures defined for man-
machine interaction in the smart home application scene, and the solution implements efficient and effective
interaction using low-cost smartwatches, and the interaction accuracy is >87%. The experiments also show
that the algorithm proposed in this paper can be well applied to the perceptual control of smart household
appliances, and has high practical value for the application design of the perceptual interaction function of
household appliances.

\end{frame}


\section[Objective]{ }
\begin{frame}{Objective}
\justifying \small 
The world advances in technology every second, most of them targetting one goal, Artificial Intelligence. For almost a decade, smart systems have played an important part in human daily life.  Our project aims to further enhance the Smart Home Experience by Deploying a Virtual Assistant. " " as it's named can do much more than just Turn lights ON/OFF. "" can also analyse the households consumption and provide insights that can help reduce the monthly Electricity Bills
\end{frame}

\section[Literature Survey]{ }
\begin{frame}{Objective}
\justifying \small 
\hskip6cm The world 
\end{frame}
\section[System Design]{ }

\begin{frame}  
\frametitle{Area of Interest} 
Our Project mainly works around,
%Points that Appear One By One
\begin{enumerate} 
	\item<1-| alert@1> IoT - \textit{Internet of Things}
	\item<2-| alert@2> AI  - \textit{Artificial Intelligence}
	\item<3-| alert@3> Data Analysis
	\item<4-| alert@4> Web/App Development  
\end{enumerate}
\end{frame}

\section[References]{}
\begin{frame}[allowframebreaks]
\frametitle{References}

\setbeamertemplate{bibliography item}[text]
\begin{thebibliography}{10}

\bibitem{nlp}
\small T. Kim, S. Bae and Y. An, "Design of Smart Home Implementation Within IoT Natural Language Interface,"\\ \textit{IEEE Access, vol. 8}, 2020, 
\\ doi: 10.1109/ACCESS.2020.2992512.

\bibitem{sw}
Y. Li et al, "Control Your Home With a Smartwatch,"\\ \textit{IEEE Access, vol. 8}, 2020,\\  doi: 10.1109/ACCESS.2020.3007328.

\bibitem{sw}
A. Parsa, T. A. Najafabadi and F. R. Salmasi,"A Hierarchical Smart Home Control System for Improving Load Shedding and Energy Consumption: Design and Implementation,"\\ \textit{IEEE Sensors Journal, vol. 19, no. 9}, 2019,
\\ doi: 10.1109/JSEN.2018.2880867.

\bibitem{sw}
Y. Li et al, "Control Your Home With a Smartwatch,"\\ \textit{IEEE Access, vol. 8}, 2020,\\ doi: 10.1109/ACCESS.2020.3007328.

\bibitem{sw}
A. Parsa, T. A. Najafabadi and F. R. Salmasi, "A Hierarchical Smart Home Control System for Improving Load Shedding and Energy Consumption: Design and Implementation,"
\\ \textit{IEEE Sensors Journal, vol. 19, no. 9}, 2019,
\\ doi: 10.1109/JSEN.2018.2880867.

\end{thebibliography}

\end{frame}
	
\section[Thank you !]{Any Questions?}


\end{document}